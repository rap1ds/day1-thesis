%
% This is a Dippa editor document template. The template suits best for academic papers
% written in Finnish.
%

% Add a document type 'article' that suits for academic papers
\documentclass[a4paper]{article}

% Encoding that fits to Finnish language
\usepackage[utf8]{inputenc}
\usepackage[T1]{fontenc}

% Hyphenation for Finnish
% \usepackage[finnish]{babel}

% References.
% NOTICE: References are currently not supported by Dippa Editor
\usepackage{natbib}

\usepackage{verbatim}

\bibpunct{(}{)}{;}{a}{,}{,}

\usepackage{fullpage}

\usepackage[top=50pt, bottom=50pt, left=70pt, right=70pt]{geometry}

% Begin the document
\begin{document}

% Document title
\title{\huge Free soda for IT-company employees - Does the increased job satisfaction and work performance cover the added costs?}
\date{\vspace{-5ex}}
\maketitle

\normalsize

% Document title
\begin{comment}
\title{\huge Free soda for IT-company employees}
\date{2.1.2013}
\author{Mikko Koski \\ mikko.koski@aalto.fi}
\maketitle
\end{comment}

\large

% First section
\section{Introduction}

\begin{comment}
Why is this research important? 
Is there a bigger phenomenon that this research of yours is part of? 
Why people in your profession should care about this thesis?
\end{comment}

Job satisfaction is one of the key elements in today's work life. From a company point-of-view an employee who enjoys the work is more effective that an employee who dislikes the work. Companies are also competing from the best employees and one key factor to draw good employee candidates is to have a working environment where employees enjoy to work.

Job satisfaction plays also a great role in how well the employee, especially the older employees cope in a work environment. In Finland, there is a pressure to raise the age of retirement and high job satisfaction has been identified to be one of the key elements.

In this paper I am going to research whether it would be reasonable for a Finnish IT-company Futurice to offer free soda for its employees. The research tries to answers whether offering free freshments would increase the job satisfactory level and performance. 

\section{Literature}

\begin{comment}
What has been done related to this (mainly in academic publications)? 
What do the authors say about the topic? 
How does your research question relate to these previous studies? 
How do you apply them or add to them? Based on what they say, what do you say?  
\end{comment}

It's common believe that caffeine has a positive effects for work performance. However, studies has shown that the perceived positive effects are almost wholly attributable to reversal of adverse withdrawal effects \citep{james2005}.



\section{Research Question}

The main research question of this paper is: \textit{is it worthwhile for an IT-company to offer free soda for employees?} 

To answer to the main research question, there are three subquestions: Does free soda have a positive effect on job satisfaction? Does it increase the work performance? How much would free soda cost to the company? Its considered that offering free soda is worthwhile for an IT-company if the increase in job satisfaction and performance overcome the costs.

\begin{comment}
Based on what other people have studied before, what is the question that no one has really answered yet? 
What is the main question, and what are perhaps the two or three sub-questions that you need to answer to be able to answer the main question? 
Be sure of what you write, because you will have to answer to this question ☺
\end{comment}

\section{Method}

\begin{comment}
How do you find an answer to the research question? 
How do you gather data? 
From where do you gather data? 
How do you analyze the data? 
Out of all the methods in the world, why did you choose this one? 
What is good about it and what is not? 
What were the alternative methods, and what were their pros and cons? 
\end{comment}

The method to answer to the research question is a formal interviews. The interviews was selected as a research method because with the interviews both the consumption of soda and the perceived effect can be answered. The interviews was decided to be formal to be able to draw some statistical analysis on the results.

The interviews were conduction in Finnish IT-company Futurice. The people selected to the interviews was random. 

Other methods that were considered for analysing soda consumption were analysing consumption from historical data and analysing consumption by setting up a video surveillance equipment next to the vending machine. However, at the time of research no historical data was available. On the other hand, due to time and equipment limitation setting up a video surveillance was excluded.

The result of the interviews about the soda consumption may have some error compared to analysing historical data or observation of the vending machine by video surveillance. However, by analysing the historical data or observing the vending machine its impossible to draw conclusions about the effects of the soda use, where as interviews can answer also to this question.

\section{Results}

Ten people answered to the questionaire. 3 of them were women and 7 were men. The average age was 38.2 years.

The results show the employees drink a very little amount of soda when they are working. Only 1 person out of 10 drank soda from company's vending machine leading to very low average soda consumption, 0.1 bottles per week. The main reason for drinking soda is thirst (7 out of 10) and taste (6 out of 10) where as getting extra energy from sugar (4 out of 10) and caffeine (3 out of 10) where not that important. From the scale of -2 to +2, people thought that when they drink a soda, it boost their work performance by +0.8.

The results also show that if the soda was free it would increase the consumption by 0.7 bottles per week (from 0.1 to 0.8). However, free soda would result in only a small increase in job satisfaction (average +0.4 in scale from -2 to 2) and work performance (average +0.3 in scale from -2 to +2).

Finally, when asked should the company offer free soda to employees, 2 answered no, 4 did not care and 4 answered that company should offer free soda.

The cost of a soda bottle is 2,1 euros in average. If the soda was free people would by 0.8 bottles per week leading to overall cost of 1,68 euros per employee per week.

The result of the study is that with an extra investment of 1,68 euros per employee per week a company can increase the job satisfaction level by 0.4 and work performance level by 0.3. Even though the increment for satisfaction and performance are minimal, it can be argued that due to the very minimal cost increase, it is worthwhile for company to offer free soda for employees.

\begin{comment}
What is the answer to the research question? 
What are the answers to the sub-questions? 
Keep this simple and clear.
\end{comment}

\section{Discussion}

The result of the paper shows that offering free soda would add only a minimal cost to the company. Even though the increase in job satisfaction and work performance by free soda is very small, due to the minimal added costs it can be argued that it is worthwhile for company to offer the free soda.

10 people from a company with more that 150 people were interviewed. 5 of the 10 people interviewed were managers, 4 were coders and 1 was user experience specialist. Even though the number of people interviewed was sufficient to form a representative sample, the ratio of manager vs. coders does not match to company's real ratio of manager and coders. In reality, the ratio of coders is a lot higher.

The results also show that people use very little soda (0.1 bottle per week). The low soda consumption and the facts that free soda has very minor effect on job satisfaction and work performance indicate that free soda is not playing a major role in overall job satisfaction and work performance.

Even though the soda consumption in the company was very low, people still think the company should be offering free soda for employees (4 of 10 answered that company should offer free soda). The structured questionnaire could not reveal all the reasons why people feel it is important that company offers the free soda. Further research could try to answers to this question by conducting a semi-structured or open interview to reveal the other reasons behind.

\begin{comment}
How could someone criticize your results? 
Are they internally valid (the data was gathered and analyzed correctly)? 
Are the results externally valid (can they be generalized and how)? 
Based on the results, what can you say about the bigger picture you described in your introduction? 
How could someone apply your results for further research? 
Or perhaps apply in a non-research context (e.g., in a company or in everyday life)?
\end{comment}

\bibliographystyle{plainnat}
\bibliography{ref}

% End of the document
\end{document}